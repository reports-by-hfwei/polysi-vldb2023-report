% si-polygraph.tex

%%%%%%%%%%%%%%%%%%%%
\begin{frame}{\polysi: Polygraph based Characterization of SI}
	\begin{center}
		% \fig{width = 0.90\textwidth}{figs/polysi-checker-polygraph}
		% \vspace{0.30cm}
		Before this, we first review the {\it dependency graph} based \\[5pt]
		characterization of SI~\ncite{AnalysingSI:JACM2018}.

		\vspace{0.30cm}
		\begin{theorem}[Theorem 4.1 of~\ncite{AnalysingSI:JACM2018}]
			Informally, a history satisfies SI if and only if \\[3pt]
			\teal{there exists} a \red{dependency graph} for it that \\[3pt]
			contains only cycles (if any) with \blue{at least two adjacent $\RW$} edges.
		\end{theorem}
	\end{center}
\end{frame}
%%%%%%%%%%%%%%%%%%%%

%%%%%%%%%%%%%%%%%%%%
\begin{frame}{Dependency Graph based Characterization of SI}
  \begin{center}
		\only<4>{
			% $T_{0} \rel{\WR} T_{A} \land T_{0} \rel{\WW} T_{B}
			%   \implies T_{A} \rel{\RW} T_{B}$
			$T_{A}$ reads from $T_{0}$ which is overwritten by $T_{B}$
		}
		\only<5>{
			% $T_{0} \rel{\WR} T_{B} \land T_{0} \rel{\WW} T_{A}
			%   \implies T_{A} \rel{\RW} T_{A}$
			$T_{B}$ reads from $T_{0}$ which is overwritten by $T_{A}$
		}
		\only<7->{
			\red{$\boxed{\text{\it Suppose that}\;\; T_{A} \rel{\WW} T_{B}}$}
		}

		\vspace{0.20cm}
		{% banking-lost-update-dep-tikz.tex

\begin{tikzpicture}[
  node distance = 0.8cm and 2.0cm,
  wr/.style = {->, thick},
  ww/.style = {->, thick, dashed, red},
  rw/.style = {->, thick, dotted, blue},
  txn/.style = {draw, inner sep = 3pt, align = center}]

  \node[txn, label = above : $T_{0}$] (t) {$\writeevent(\acct, 0)$};

  \node[txn, label = above : $T_{A}$, above right = of t] (t-alice)
    {$\readevent(\acct, 0)$ \\[2pt] $\writeevent(\acct, 50)$};
  \node[txn, label = below : $T_{B}$, below right = of t] (t-bob)
    {$\readevent(\acct, 0)$ \\[2pt] $\writeevent(\acct, 25)$};

  \node[txn, label = above : $T'_{A}$, right = 6.0cm of t] (t-alice-read) {$\readevent(\acct, 25)$};

  \uncover<2->{
    \draw[wr, sloped] (t) to node[below]{$\WR$} (t-alice.west);
    \draw[wr, sloped] (t) to node[above]{$\WR$} (t-bob.west);
    \draw[wr, sloped] (t-bob.east) to node[below]{$\WR$} (t-alice-read.south);
  }

  \uncover<3->{
    \draw[ww, bend left, sloped] (t) to node[above]{$\WW$} (t-alice);
    \draw[ww, bend right, sloped] (t) to node[below]{$\WW$} (t-bob);
  }
  \uncover<4->{
    \draw[rw, bend right = 60] (t-alice.-145) to node[]{$\RW$} (t-alice.-145 |- t-bob.north);
  }
  \uncover<5->{
    \draw[rw, bend right = 60] (t-bob.35) to node[]{$\RW$} (t-bob.35 |- t-alice.south);
  }
  \uncover<7->{
    \draw[ww] (t-alice) to node[]{$\WW$} (t-bob);
  }
\end{tikzpicture}}
		\vspace{0.20cm}

		\only<2>{
			$\WR$: ``write-read'' dependency capturing the ``read-from'' relation
		}
		\only<3>{
			$\WW$: ``write-write'' dependency capturing the version order on $\acct$
		}
		\only<4-5>{
			$\RW$: ``read-write'' dependency
		}
		\only<6>{
			The cycle $T_{A} \rel{\RW} T_{B} \rel{\RW} T_{A}$
			is \blue{allowed} by SI.
		}
		\only<7>{
			undesired cycle for SI: $T_{A} \rel{\WW} T_{B} \rel{\RW} T_{A}$
		}
  \end{center}
\end{frame}
%%%%%%%%%%%%%%%%%%%%

%%%%%%%%%%%%%%%%%%%%
\begin{frame}{Dependency Graph based Characterization of SI}
	\begin{center}
		\uncover<2->{
			\red{$\boxed{\text{\it Suppose that}\;\; T_{B} \rel{\WW} T_{A}}$}
		}

		\vspace{0.20cm}
		{\input{tikz/banking-lost-update-depgraph-ww-tbta-tikz}}
		\vspace{0.20cm}

		\uncover<2>{
			undesired cycle for SI: $T_{B} \rel{\WW} T_{A} \rel{\RW} T_{B}$
		}
	\end{center}
\end{frame}
%%%%%%%%%%%%%%%%%%%%

%%%%%%%%%%%%%%%%%%%%
\begin{frame}{Dependency Graph based Characterization of SI}
  \begin{center}
		We have considered both bases $T_{A} \rel{\WW} T_{B}$
		and $T_{B} \rel{\WW} T_{A}$, \\[2pt]
		and each case leads to an undesired cycle for SI.

		\vspace{0.20cm}
		\fig{width = 0.65\textwidth}{figs/banking-lost-update-wr}
		\vspace{0.20cm}

		Therefore, this history does not satisfy SI.
  \end{center}
\end{frame}
%%%%%%%%%%%%%%%%%%%%

%%%%%%%%%%%%%%%%%%%%
\begin{frame}{Dependency Graph based Characterization of SI}
	\begin{theorem}[Equivalence of Theorem 4.1 of~\ncite{AnalysingSI:JACM2018}]
		Informally, a history satisfies SI if and only if \\[3pt]
		\teal{there exists} a \red{dependency graph} $\G$ for it such that \\[3pt]
		\cyan{the induced graph of $\G$} ${\boxed{((\SO_{\G} \cup \WR_{\G} \cup \WW_{\G}) \comp \RW_{\G}?)}}$ \text{\it is acyclic}.
	\end{theorem}
\end{frame}
%%%%%%%%%%%%%%%%%%%%

%%%%%%%%%%%%%%%%%%%%
\begin{frame}{Dependency Graph based Characterization of SI}
	\[
		\text{\it induced graph}\; {\boxed{((\SO_{\G} \cup \WR_{\G} \cup \WW_{\G}) \comp \RW_{\G}?)}} \text{\it\; for $\G$}
	\]

	\begin{center}
		\resizebox{0.60\textwidth}{!}{\input{tikz/banking-lost-update-depgraph-theorem-tikz.tex}}
		\vspace{0.30cm}

		\uncover<2->{
			first composing ($\comp$) the $\SO$/$\WR$/$\WW$ edges with the $\RW$ edges
		}

		\vspace{0.20cm}
		\uncover<3>{
			then deleting all the $\RW$ edges
		}
	\end{center}
\end{frame}
%%%%%%%%%%%%%%%%%%%%

%%%%%%%%%%%%%%%%%%%%
\begin{frame}{Polygraph: A Family of Dependency Graphs}
	\begin{center}
		Consider the two cases of $\WW$ dependencies between $T_{A}$ and $T_{B}$.
	\end{center}

	\vspace{-0.20cm}
	\begin{columns}[c]
		\column{0.50\textwidth}
			\fig{width = 0.80\textwidth}{figs/banking-lost-update-depgraph}
		\column{0.50\textwidth}
			\fig{width = 0.80\textwidth}{figs/banking-lost-update-depgraph-ww-tbta}
	\end{columns}

	\vspace{-0.20cm}
	\begin{center}
		\pause
		\fig{width = 0.50\textwidth}{figs/banking-lost-update-polygraph}
		polygraph:
		$\tuple{\cyan{\eithervar} \triangleq \set{T_{A} \rel{\WW} T_{B}},
				\cyan{\orvar} \triangleq \set{T_{B} \rel{\WW} T_{A}, T_{A}' \rel{\RW} T_{A}}}
		$
	\end{center}
\end{frame}
%%%%%%%%%%%%%%%%%%%%

%%%%%%%%%%%%%%%%%%%%
\begin{frame}{\textsc{PolySI}: A Running Example}
	\begin{center}
		To explain the whole \polysi{} procedure with a running example.

		\vspace{0.50cm}
		\fig{width = 0.90\textwidth}{figs/polysi-checker-pruning-encoding-solving}
	\end{center}
\end{frame}
%%%%%%%%%%%%%%%%%%%%

%%%%%%%%%%%%%%%%%%%%
\begin{frame}{\textsc{PolySI}: A Running Example}
	\begin{center}
		\resizebox{0.90\textwidth}{!}{\input{tikz/polysi-alg-tikz}}

		\only<1>{
			$\WW$ between $T_{0}$, $T_{1}$, and $T_{5}$ (on $\keyxvar$)
			and between $T_{0}$ and $T_{2}$ (on $\keyyvar$)
		}

		\only<2>{
			The $T_{5} \rel{\WW(\keyxvar)} T_{0}$ case is pruned
			due to $T_{0} \rel{\SO} T_{5} \rel{\WW(\keyxvar)} T_{0}$.
		}

		\only<4>{
			The $T_{0} \rel{\WW(\keyxvar)} T_{5}$ case becomes known.
		}

		\only<6>{
			The $T_{1} \rel{\WW(\keyxvar)} T_{0}$ case is pruned
			due to $T_{3} \rel{\RW(\keyxvar)} T_{0} \rel{\WR(\keyyvar)} T_{3}$.
		}

		\only<8>{
			The $T_{0} \rel{\WW(\keyxvar)} T_{1}$ case becomes known.
		}

		\only<10>{
			The $T_{2} \rel{\WW(\keyyvar)} T_{0}$ case is pruned, \\
			while the $T_{0} \rel{\WW(\keyyvar)} T_{2}$ case becomes known.
		}
		\only<12>{
			The $\WW$ order between $T_{1}$ and $T_{5}$ is still uncertain after pruning.
		}
	\end{center}
\end{frame}
%%%%%%%%%%%%%%%%%%%%

%%%%%%%%%%%%%%%%%%%%
\begin{frame}{\textsc{PolySI}: A Running Example}
	\vspace{-0.50cm}
	\[\tuple{
		\uncover<2->{\purple{\eithervar} = \set{T_{1} \rel{\WW(\keyxvar)} T_{5},
			T_{3} \rel{\RW(\keyxvar)} T_{5}}},
		\uncover<2->{\violet{\orvar} = \set{T_{5} \rel{\WW(\keyxvar)} T_{1}}}
	}\]

	\vspace{-0.30cm}
	\begin{center}
		\resizebox{0.80\textwidth}{!}{\input{tikz/polysi-alg-encoding-tikz}}
	\end{center}
	\vspace{-0.80cm}

	\uncover<3->{
		\[
			\underbrace{\purple{(\BV_{1,5} \land \BV_{3,5} \land \lnot \BV_{5,1})}}_{\purple{\eithervar}} \lor
			\underbrace{\violet{(\BV_{5,1} \land \lnot \BV_{1,5} \land \lnot \BV_{3,5})}}_{\violet{\orvar}}
		\]
	}
\end{frame}
%%%%%%%%%%%%%%%%%%%%

%%%%%%%%%%%%%%%%%%%%
\begin{frame}{\textsc{PolySI}: A Running Example}
	\vspace{-0.50cm}
	\[
		\cyan{\text{induced graph}}\; \cyan{\mathcal{I}}\;
		  {\boxed{((\SO_{\G} \cup \WR_{\G} \cup \WW_{\G}) \comp \RW_{\G}?)}}
	\]

  \begin{columns}
		\column{0.50\textwidth}
			\fig{width = 1.00\textwidth}{figs/polysi-alg-final}
		\column{0.50\textwidth}
			\fig{width = 1.00\textwidth}{figs/polysi-alg-encoding}
	\end{columns}

	\vspace{0.50cm}
	\uncover<2->{
	\[
		T_{1} \rel{\WR} T_{3} \rel{\RW} T_{5}:\;
		  \BV_{1,5}^{\cyan{\;\mathcal{I}}} = \BV_{1,3} \;\land\; \BV_{3,5} \quad
	\]

	\vspace{-0.20cm}
	\begin{center}
		The presence of the edge $T_{1} \to T_{5}$ in the induced graph $\mathcal{I}$ \\[2pt]
		depends on that of the edges $T_{1} \to T_{3}$ and $T_{3} \to T_{5}$ in the polygraph.
	\end{center}
	}
	% \vspace{-0.30cm}
	% \uncover<2->{
	% \[
	% 	T_{1} \rel{\WR} T_{3} \rel{\RW} T_{2}:\;
	% 	  \BV_{1,2}^{\cyan{\;\mathcal{I}}} = \BV_{1,3} \;\land\; \BV_{3,2} \quad
	% \]
	% }
\end{frame}
%%%%%%%%%%%%%%%%%%%%

%%%%%%%%%%%%%%%%%%%%
\begin{frame}{\textsc{PolySI}: A Running Example}
	\begin{center}
		Feed the SAT formula into the \blue{MonoSAT} solver~\ncite{MonoSAT:AAAI2015} \\[2pt]
		which is optimized for \purple{\it cycle detection}.

		\vspace{0.20cm}
		\fig{width = 0.40\textwidth}{figs/sat-solver}
	\end{center}
\end{frame}
%%%%%%%%%%%%%%%%%%%%

%%%%%%%%%%%%%%%%%%%%
\begin{frame}{\textsc{PolySI}: A Running Example}
	\begin{center}
		\fig{width = 0.50\textwidth}{figs/polysi-alg-cycle}

		\vspace{0.20cm}
		The MonoSAT solver finds an undesired cycle for SI.
	\end{center}
\end{frame}
%%%%%%%%%%%%%%%%%%%%