% intro.tex

%%%%%%%%%%%%%%%%%%%%
\begin{frame}{Database Transactions}
  \begin{center}
    A database transaction is a \blue{\it group} of operations \\[10pt]
    \figdouble{0.40}{0.80}{figs/all-or-none}{0.40}{0.80}{figs/all-or-none}
    that should be executed \red{atomically}.
  \end{center}
\end{frame}
%%%%%%%%%%%%%%%%%%%%

%%%%%%%%%%%%%%%%%%%%
\begin{frame}{Isolation Levels}
  \begin{center}
    Transactions may be executed concurrently. \\[6pt]
    The isolation levels specify how they are isolated from each other.
    \vspace{0.60cm}
  \end{center}
\end{frame}
%%%%%%%%%%%%%%%%%%%%

%%%%%%%%%%%%%%%%%%%%
\begin{frame}{Serializability (SER)}
  \begin{center}
    All transactions appear to execute serially, one after another.

    \vspace{0.60cm}
    too expensive, especially for distributed transactions
  \end{center}
\end{frame}
%%%%%%%%%%%%%%%%%%%%

%%%%%%%%%%%%%%%%%%%%
\begin{frame}{Snapshot Isolation (SI)}
  \begin{center}
    \fig{width = 0.90\textwidth}{figs/si-hierarchy}
  \end{center}
\end{frame}
%%%%%%%%%%%%%%%%%%%%

%%%%%%%%%%%%%%%%%%%%
\begin{frame}{Snapshot Isolation (SI)}
  \begin{center}
    example
    % \fig{width = 0.80\textwidth}{figs/si-example}
    \vspace{0.60cm}
    \begin{description}[Snapshot Write:]
      \item[Snapshot Read:] Each transaction reads data from a snapshot
        of committed data valid as of the (logical) time the transaction started.
      \item[Snapshot Write:] Concurrent transactions cannot write to the same key.
        One of them must be aborted.
    \end{description}
  \end{center}
\end{frame}
%%%%%%%%%%%%%%%%%%%%

%%%%%%%%%%%%%%%%%%%%
\begin{frame}{SI Prevents the ``Lost Update'' Anomaly}
  \begin{center}
    % lost-update.tex

\begin{tikzpicture}[
  node distance = 0.8cm and 2.0cm,
  txn/.style = {draw, inner sep = 3pt, align = center}]

  \node[txn, label = above : $T_{0}$] (t) {$\writeevent(\acct, 0)$};

  \uncover<2->{
  \node[txn, label = above : $T_{A}$, above right = of t] (t-alice)
    {$\readevent(\acct, 0)$ \\[2pt] $\writeevent(\acct, 50)$};
  }
  \uncover<3->{
  \node[txn, label = below : $T_{B}$, below right = of t] (t-bob)
    {$\readevent(\acct, 0)$ \\[2pt] $\writeevent(\acct, 25)$};
  }

  \uncover<4->{
    \node[txn, label = above : $T'_{A}$, right = 6.0cm of t] (t-alice-read) {$\readevent(\acct, 25)$};
  }
\end{tikzpicture}

    \uncover<3->{
      \vspace{0.50cm}
      $T_{A}$ and $T_{B}$ are executed concurrently.
    }
  \end{center}
\end{frame}
%%%%%%%%%%%%%%%%%%%%

%%%%%%%%%%%%%%%%%%%%
\begin{frame}{SI Prevents the ``Causality Violation'' Anomaly}
  \begin{center}
    % post-causality-violation-tikz.tex

\begin{tikzpicture}[
  node distance = 1.5cm and 2.0cm,
  wr/.style = {->, thick},
  ww/.style = {->, thick, dashed},
  rw/.style = {->, thick, dotted},
  txn/.style = {draw, inner sep = 3pt, align = center}]

  \node[txn, label = left : $T_{A}$] (t-alice) {$\writeevent(\keyxvar, \post)$};
  \uncover<2->{
  \node[txn, label = left : $T_{B}$, below = of t-alice] (t-bob)
    {$\readevent(\keyxvar, \post)$ \\[2pt] $\writeevent(\keyyvar, \comment)$};
  }
  \uncover<3->{
  \node[txn, label = left : $T_{C}$, below = of t-bob] (t-carl)
    {$\readevent(\keyxvar, \emptypost)$ \\[2pt] $\readevent(\keyyvar, \comment)$};
  }
\end{tikzpicture}
  \end{center}
\end{frame}
%%%%%%%%%%%%%%%%%%%%

%%%%%%%%%%%%%%%%%%%%
\begin{frame}{SI Allows the ``Write Skew'' Anomaly}
  \begin{center}
    % \input{tikz/lost-update-tikz}
  \end{center}
\end{frame}
%%%%%%%%%%%%%%%%%%%%

%%%%%%%%%%%%%%%%%%%%
\begin{frame}{Databases that Claim to Support SI}
  \begin{center}
    database logos
  \end{center}
\end{frame}
%%%%%%%%%%%%%%%%%%%%

%%%%%%%%%%%%%%%%%%%%
\begin{frame}{Snapshot Isolation (SI)}
  \begin{center}
    Database systems may fail to provide SI as they claim.

    +papers
  \end{center}
\end{frame}
%%%%%%%%%%%%%%%%%%%%

%%%%%%%%%%%%%%%%%%%%
\begin{frame}{The SI Checking Problem}
  \begin{definition}[The SI Checking Problem]
    The SI checking problem is the \purple{decision problem} of \\[5pt]
    determing whether a given \teal{history $\H = (T, \SO)$} satisfies SI?
  \end{definition}

  \fig{width = 0.60\textwidth}{figs/si-checking}
\end{frame}
%%%%%%%%%%%%%%%%%%%%

%%%%%%%%%%%%%%%%%%%%
\begin{frame}{Motivation: Black-box SI Checker}
  \begin{center}
    Since the internals of database systems are often unavailable
    or are hard to understand, \\[6pt]
    a \blue{\it black-box} SI checker is highly desirable.
  \end{center}
\end{frame}
%%%%%%%%%%%%%%%%%%%%

%%%%%%%%%%%%%%%%%%%%
\begin{frame}{Motivation: Black-box SI Checker}
  \begin{center}
    % checker-tikz.tex

\begin{tikzpicture}[
  node distance = 0.5cm and 1.0cm,
  every label/.style = {font = \normalsize}]

	\node[draw, thick, inner sep = 8pt] (checker) {SI Checker};

	\coordinate (anchor) at ($(checker.west) + (-2.5, 0)$);
	\draw[->, thick] (anchor) to
	  node[above]{A history}
	  node[below]{$\H = (\T, \SO)$}
		(checker);

	\node[above right = of checker] (sat) {\yes};
	\node[below right = of checker] (unsat) {\no};

	\draw[->, thick] (checker) to (sat);
	\draw[->, thick] (checker) to (unsat);

	\uncover<1->{
		\node[above = 0.30cm of checker, blue] (sound) {\it Sound};
	}
	\uncover<2->{
		\node[below = 0.30cm of checker, blue] (complete) {\it Complete};
	}
	\uncover<3->{
		\node[right = 0.30cm of checker, blue] (efficient) {\it Efficient};
	}
	\uncover<4->{
		\node[below = 0.00cm of unsat, blue] (informative) {\it Informative};
	}
\end{tikzpicture}

    \vspace{0.30cm}
    \only<2>{
      \blue{\it Sound:} If the checker says \no, then the history is not SI.
    }
    \only<3>{
      \blue{\it Complete:} If the checker says \yes, then the history is SI.
    }
    \only<4>{
      \blue{\it Efficient:} The checker should scale up to large workloads.
    }
    \only<5>{
      \blue{\it Informative:} The checker should provide understandable counterexamples
        if it says \no.
    }
  \end{center}
\end{frame}
%%%%%%%%%%%%%%%%%%%%

%%%%%%%%%%%%%%%%%%%%
\begin{frame}{Motivation: Black-box SI Checker}
  related-work
\end{frame}
%%%%%%%%%%%%%%%%%%%%

%%%%%%%%%%%%%%%%%%%%
\begin{frame}{Contributions: the \textsc{PolySI} Checker}
  \begin{center}
    % polysi-checker-tikz.tex

\begin{tikzpicture}[
  node distance = 0.5cm and 1.0cm,
  every label/.style = {font = \normalsize}]

	\node[draw, thick, inner sep = 8pt, fill = yellow!50] (polysi) {\textsc{PolySI}};

	\coordinate (anchor) at ($(polysi.west) + (-2.5, 0)$);
	\draw[->, thick] (anchor) to
	  node[above]{A history}
	  node[below]{$\H = (\T, \SO)$}
		(polysi);

	\node[above right = of polysi] (sat) {\yes};
	\node[below right = of polysi] (unsat) {\no};

	\draw[->, thick] (polysi) to (sat);
	\draw[->, thick] (polysi) to (unsat);

	\uncover<2->{
		\node[draw, below right = 1.50cm and 0.30cm of polysi.south, fill = brown!50, inner sep = 5pt, align = center]
			(monosat) {MonoSAT \\ Solving};
	}
	\uncover<1->{
		\node[draw, below left = 1.50cm and 0.30cm of polysi.south, fill = brown!50, inner sep = 5pt, align = center]
			(encoding) {SAT \\ Encoding};
	}
	\uncover<3->{
		\node[draw, left = 0.60cm of encoding, fill = brown!50, inner sep = 5pt, align = center]
			(pruning) {Constraints \\ Pruning};
	}
	\uncover<4->{
		\node[draw, right = 0.60cm of monosat, fill = brown!50, inner sep = 5pt, align = center]
			(counterexamples) {Counterexamples \\ Extracting};
	}
	\uncover<3->{
		\draw[->, thick, brown] (pruning) to (encoding);
	}
	\uncover<2->{
		\draw[->, thick, brown] (encoding) to (monosat);
	}
	\uncover<4->{
		\draw[->, thick, brown] (monosat) to (unsat);
		\draw[->, thick, brown] (unsat) to (counterexamples);
	}

	\uncover<1->{
		\draw[dashed, very thick, blue] (polysi.south west) to (pruning.north west);
		\draw[dashed, very thick, blue] (polysi.south east) to (counterexamples.north east);
	}
\end{tikzpicture}

    \vspace{0.50cm}
    \only<2>{
      \blue{{\it Sound} \&{\it Complete:}}
        polygraph-based characterization of SI
    }
    \only<3>{
      \blue{\it Efficient:} utilizing MonoSAT solver optimized for graph problems
    }
    \only<4>{
      \blue{\it Efficient:} domain-specific pruning before encoding
    }
    \only<5>{
      \blue{\it Informative:} extract counterexamples from the unsatisifiable core
    }
  \end{center}
\end{frame}
%%%%%%%%%%%%%%%%%%%%

%%%%%%%%%%%%%%%%%%%%
\begin{frame}{Contributions: PolySI}
  \begin{center}
    PolySI found SI violations in production database systems.

    \vspace{0.80cm}
    PolySI outperformed state-of-the-art black-box SI checkers
    and scales up to large workloads.
  \end{center}
\end{frame}
%%%%%%%%%%%%%%%%%%%%